%%%%%%%%%%%%%%%
% kolize

\section{Kolize}

Bez kolizí se neobejde téměř žádná fyzikální simulace.
Například částicový systém díky kolizím interaguje s okolím.
Kolize bývají složité na výpočet a proto je potřeba algoritmy zefektivňovat.
Jedním z vylepšení algoritmů detekcí kolizí jsou obalová tělesa.
Dalším vylepšením může být hierarchické rozdělení scény.

V intru jsou kolize počítány jen pro částicové a elastické systémy.
Tyto systémy kolidují s geometrií jeskyně a tím se odráží od jejího povrchu.
Pro jednoduchost budeme kolize počítat jen pro body.
Abychom zjistili, zda je bod v kolizi s jeskyní, musíme zjistit, se kterým trojúhelníkem bod koliduje.
Tento přístup by nebyl příliš rychlý a implementace algoritmu by zabrala příliš místa.
Místo toho použijeme jednoduchý způsob.
Souřadnice bodu, pro který počítáme kolizi, můžeme použít jako index do volumetrické reprezentace jeskyně.
Pokud hodnota na souřadnicích bodu překročí určitý práh (stejný práh je použit i pro algoritmus Marching Tetrahedra) je v daném místě kolize.

Nejprve jsem navrhl obecný algoritmus, který dokázal získat hodnotu pro neceločíselné souřadnice z $d$ dimenzionálního pole.
Tento algoritmus byl rekurzivní, relativně malý a díky své obecnosti znovupoužitelný.
Nicméně nebyl příliš rychlý.
Proto jsem navrhl algoritmus, který je rychlejší, ale pracuje jen pro trojrozměrné pole.
$P_0=(p_0,p_1,p_2)$ je bod, pro který počítáme kolizi.
Nejprve přesuneme bod do rozsahu $r=(r_0,r_1,r_2)$ (šířka, výška, délka) podle vztahu: $P_1=((P_0 \% r) + r) \% r$.
Operace modulo $\%$ pracuje po složkách vektoru.
Poté získáme nejvyšší celočíselný index $I=(i_0,i_1,i_2)=\lfloor P_1 \rfloor$, který je menší než $P_1$.
Rozdíl $v=(v_0,v_1,v_2)=P_1-I$ udává poměr míchání hodnot na indexech o jedničku vyšších než $I$ v některé ose.
Hodnota $h(P_0)$ trojrozměrného pole v bodě $P_0$ je dána vztahem:
\begin{eqnarray*}
\label{eq:kolid}
h_0    &=& (1-v_0)h(I)      + v_0h(I_{x}) \\
h_1    &=& (1-v_0)h(I_y)    + v_0h(I_{yx}) \\
h_2    &=& (1-v_0)h(I_z)    + v_0h(I_{zx}) \\
h_3    &=& (1-v_0)h(I_{zy}) + v_0h(I_{zyx}) \\
h_{01} &=& (1-v_1)h_0       + v_1h_1 \\
h_{23} &=& (1-v_1)h_2       + v_1h_3 \\
h(P_0) &=& (1-v_2)h_{01}    + v_2h_{23}
\end{eqnarray*}
Dolní sufix indexu $I$ udává, ke které složce indexu $I$ je přičtena jednička.
Například pro index $I_{zx}=(i_0+1,i_1,i_2+1)$.

Abychom mohli správně reagovat na kolizi, musíme znát i normálu povrchu.
Normálu povrchu máme uloženou také v trojrozměrném poli, proto můžeme použít stejný algoritmus.
Reakce na kolizi spočívá v přesunutí bodu na povrch jeskyně a otočení jeho rychlosti podle normály.
Úhel dopadu se přitom musí rovnat úhlu odrazu.
Výslednou rychlost ještě zmenšíme o ztráty.
Ovlivnění rychlosti u částicových systému je přímočaré.
U elastických systémů si ale musíme dát pozor.
Ovlivňujeme rychlosti uzlů externě, mimo výpočetní model elastického systému.
Toto může potenciálně vést ke vzniku nestability.

