%%%%%%%%%%%%%%%%%%%%
% castice

\section{Částicové systémy}

Částicové systémy se v počítačové grafice používají k nejrůznějším účelům.
Pomocí čá\-sti\-co\-vých systémů reprezentujeme fyzikální jevy, které by se jinou cestou špatně vizualizovaly nebo simulovaly.
Jedním z příkladů může být simulace sněžení.
Další příklady jsou písek, ohňostroj nebo voda.
Částicový systém je složen s velkého množství samostatných částic.
Každá částice se chová samostatně podle určitých pravidel.
Částice mají svoji pozici a rychlost.
Tyto parametry slouží k popisu pohybu.
Rovnice pro popis pohybu jsou \ref{eq:pozice1}, \ref{eq:rychlost1} a \ref{eq:newton} a blíže si je popíšeme v sekci \ref{sec:elastic} o elastických systémech.

Částice budeme vykreslovat jako plošky s nanesenou texturou.
Budeme rozlišovat dva druhy.
Jedny částice se neustále natáčejí ke kameře.
Tento druh částic budeme používat pro vykreslení vodopádu, bublinek ve vodě a listů liánové rostliny.
Druhý druh si udržuje svoji orientaci.
V intru jej budeme používat pro vykreslení chomáče rostlin.
Mimo pozice a rychlosti má u sebe částice také čas.
Pokud čas částice překročí určitou hodnotu, částice zanikne.
U částicového systému si budeme definovat i emitor.
V emitoru se vytváří částice.
Pokud částice zanikne objeví se nová s počátečním nastavením v emitoru.




