%%%%%%%%%%%%%%%%%%%%
% zaver

\chapter{Závěr}
Část textu jsem převzal s drobnými úpravami ze semestrálního projektu.
Jedná se o části až po sekci Generování geometrie \ref{sec:geometrie}.
Text semestrálního projektu prošel úpravami a byl přepsán do systému Latex.
Celková velikost intra nepřesáhla 64kB.
Intro obsahuje 4 scény.
Každá z nich je ozvučená hudbou, kterou jsem složil.

V průběhu projektu mne napadlo velké množství rozšíření a vylepšení.
Některé jsem implementoval a na některé nezbyl čas.
Možným rozšířením intra by mohlo být přidání statických a dynamických stínů.
Jiné rozšíření by spočívalo v implementaci SSAO ({\it screen space ambient occlusion}).
Další možností by mohlo být generování geometrie v průběhu animace.
Naimplementovat Marching Tertahedra algoritmus v shader programu OpenGL.
Naimplementovat jiné šumy.
Příkladem může být Simplex Noise.
Tento šum implementovat v shader programu.
Vhodná by mohla být implementace šumu, který netrpí aliasing efektem.
Připojit dynamiku rigidních těles a spojit ji s dynamikou elastických těles.
Přidat složitější detekci kolizí.
Převést výpočet elastických systému do OpenCL a tím jej urychlit.
Přesunout časování do zvláštního vlákna.
Vyzkoušet další numerické metody pro řešení diferenciálních rovnic.
Například implicitní Eulerovu metodu nebo metodu s využitím Taylorova rozvoje.
Rozšíření projektu z jiného úhlu pohledu by mohlo spočívat v au\-to\-ma\-ti\-zo\-va\-ném generování hudby.
%Vylepšení synchronizace hudby a obrazu.
Vytvoření algoritmů pro generování zvuků.
Přidání živých organizmů s umělou inteligencí: ryby ve vodě, pavouk, poletující mušky...
%Další zcela jiné rozšíření je využít pro vykreslování metodu raytracing implementovanou v shader programu.
Místo rozšiřování intra by bylo možné některé vlastnosti oddělit a vyvíjet je jako samostatný projekt.
Napadlo mě několik takových projektů.
Efektivní generování textur v shader programu.
Fyzikální engine.
Prozkoumání $d$ dimenzionální alternativy algoritmu Marching Tetrahedra.

V průběhu projektu jsem se naučil hodně nových věcí a vyzkoušel několik různých alternativních řešení problému.
Příkladem nechť je algoritmus Marching tetrahedra, kterým jsem nahradil Marching cubes.
Další příkladem je implementace numerické metody Runge Kutta, která nahradila Eulerovu metodu.
Implementoval jsem dobrý základ pro generování textur, který je dostatečně obecný, aby byl znovupoužitelný.
Implementoval jsem generování Voroného diagramu a šumu obecné dimenze.
Vytvořil jsem funkce, které dokáži s $d$ dimenzionálními daty pracovat.
%Příkladem je $d$ rozměrná konvoluce.
Vytvořil jsem obecný elastický a částicový systém.
Rozšířil jsem si svoje znalosti jazyka GLSL a správy velkého projektu jako takového.
Během práce na projektu jsem řešil řadu problémů v podobě překlepů nebo použití nevhodné metody.
Největší problém ale byl ukončit tvůrčí činnost v určitém bodě a projekt dokončit.
Napadalo mne hodně vylepšení a rozšíření a bylo těžké se rozhodnout je již do projektu nezahrnovat.
Přes všechny problémy jsem se při vyvíjení projektu bavil a co je nejdůležitější - rozšířil jsem si znalosti.

