%%%%%%%%%%%%%%
% glsl

\section{GLSL}

V intru jsem použil sedm shader programů.
Tři z nich používají i geometry shader.
Jsou to shader programy využívané částicovými systémy.
O vytvoření částice (plošky) se stará geometry shader.
Pro částicové systémy je vyhrazen vlastní shader program.
Pomocí uniformu lze nastavit, jestli se částice natáčí ke kameře či udržuje svoji orientaci.

Shader program, který v intru používáme pro vykreslení jeskyně má nejsložitější část fragment shader.
Jeskyně obsahuje poměrně malé množství trojúhelníků a přesto je nej\-slo\-ži\-těj\-ší na vykreslení.
Snížením počtu trojúhelníků bychom složitost nesnížili.
Jednou možností je optimalizovat fragment shader.
Vhodnou optimalizací může být minimalizace větvení programu použitím vestavěných funkcí.
Příklad větvení:
\begin{verbatim}
if (red_flag == 1){
  color = vec3(1,0,0);
}else{
  color = texture(wall,coord);
}
\end{verbatim}
Výše uvedený úsek programu obsahuje větvení pomocí uniformu $red\_flag$.
Můžeme se větvení zbavit využitím vestavěné funkce $mix$.
\begin{verbatim}
color=mix(vec3(1,0,0),texture(wall,coord),red_flag);
\end{verbatim}
Další funkce, které se dají použít pro odstranění větvení jsou $min,max,clip,clamp$ a další.
Jinou možností je omezení používání množství normalizační funkce: $normalize$.
Urychlení vykreslení scény se složitým fragment shaderem můžeme udělat i jinak.
Urychlení spočívá v omezení množství fragmentů, které musíme vykreslit.
Při vykreslování se stává, že jsou trojúhelníky kresleny ve špatném pořadí.
Vykreslujeme některé fragmenty, které jsou později přepsány novými hodnotami.
Řešení může existovat v algoritmu řazení trojúhelníků.
Potom stačí vykreslovat od nejbližších trojúhelníků.
V případě, že je nějaká část trojúhelníku zakryta, je vykreslování zastaveno hloubkovým testem.
Řazení trojúhelníků je ale složité na výpočet a zbytečně by nám zabralo místo v programu.
Jeskyně ale obsahuje málo trojúhelníků, proto vykreslíme nejprve hloubkový buffer s vypnutým zápisem barvy.
Tím se přeskočí část s fragment shaderem.
Poté vykreslíme scénu znovu, tentokrát již s povoleným zápisem barvy.
Tímto zajistíme, že se výpočet fragment shaderu provádí jen pro nezbytně nutné množství fragmentů.

