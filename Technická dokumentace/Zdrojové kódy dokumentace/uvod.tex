%========================
%UVOD

\chapter{Úvod}
Cílem této práce je vytvoření grafického intra s omezenou velikostí.
Práce pojednává o technikách minimalizace velikosti spustitelného souboru pomocí nastavení překladu, komprimování výsledného souboru, vhodného programátorského stylu a generování obsahu podle šablon.
Zabývá se generováním grafického obsahu jako jsou textury a geometrie.
V práci je popsán způsob generování vícerozměrného šumu.
Dále se zabývá vytvořením vícerozměrných Voroného diagramů.
Další část práce se zabývá tvorbou textur pomocí šumu, Voroného diagramů, barevných přechodů a transformačních operací.
Následuje sekce o generování geometrie intra.
V této části je popsán převod objemové reprezentace tělesa na povrchovou reprezentaci.
Spadají sem algoritmy Marching cubes a Marching tetrahedra.
Dále sem patří vyhlazení povrchu a optimalizace ukládání dat produkovaných těmito algoritmy.
Poté se práce zabývá bump mappingem, texturováním, tvorbou skyboxu, částicovými a elastickými systémy, vodou, terénem a pohybem kamery.

\section{OpenGL}
OpenGL je v práci využito pro vykreslování veškeré grafiky.
OpenGL („Open Graphics Library“) je nízkoúrovňová grafická knihovna.
Za vytvořením knihovny stojí společnost Silicon Graphics, Inc., jež ji uvedla v roce 1992.
Knihovna od počátku vyšla v několika verzích, přičemž verze jsou zpětně kompatibilní.
Obsahuje stovky funkcí umožňujících specifikaci a manipulaci s grafickými daty.
V dřívějších dobách byla kreslící pipeline fixní a nebylo tedy možné ji jakkoliv upravit.
Od verze OpenGL 2.0 je k dispozici jazyk GLSL, kterým lze kreslicí pipeline naprogramovat a změnit její chování.
%V práci je jazyk GLSL využíván pro různé grafické efekty.

\section{Intro}
Grafické intro bývá krátké video vytvořené programem.
Video vytvořené aplikací nebývá nijak interaktivní.
Důležitou charakteristikou programu je jeho malá velikost.
Existuje několik kategorií velikostí od 1KB přes 64KB až po neomezenou velikost.
Program obvykle nepotřebuje ke svému běhu žádná externí data, jako jsou obrázky textur a soubory s modely či hudbou.
Klíčovým prvkem intra je kontrast mezi malou velikostí programu a množstvím a silou účinku prezentovaných grafických a zvukových informací.
Tento kontrast většinou poukazuje na schopnosti daného programátora nebo skupiny, kteří intro vytvořili.
U větších skupin vznikají i různé podpůrné nástroje například pro komponování a přehrávání hudby.
Zdrojové kódy programů bývají většinou tajné a autoři si je bedlivě střeží.
Intra se objevují nejčastěji jako programy vytvořené pro zábavu či soutěž.
Vyskytují se ale i jako komerční prezentace.

Obsah intra může být krátký příběh, předvedení myšlenky, prezentace schopností autorů, ukázání svých uměleckých schopností, propagace k filmu, reklama nebo jiné věci.
Pokud se jedná o intro vytvořené pro zábavu (nejčastější případ) není obsah intra nikterak svázán a náplň je zcela v rukou autora a záleží jen na něm, co bude jeho intro zobrazovat.
Časté jsou abstraktní scény, které nemají mnoho společného s realitou, poskytují však vydatný grafický zážitek.
Abstraktní intra se nejčastěji objevují u velmi malých velikostí programů.
Nevýhodou těchto typů inter bývá, že ač jsou grafické efekty sebelepší časem, s vývojem grafického hardwaru, stejně zastarávají.
Proto bývá důležitým prvkem intra příběh.
Silný příběh může hodnotu intra značně zvýšit. Nicméně to platí i naopak.

Grafická intra jsou fenoménem.
Internet jich obsahuje tisíce a komunita kolem inter je velmi živá.
Pořádají se různé soutěže s oceněními.
Do budoucna lze očekávat, že tvorba inter bude pokračovat.
S přibývajícím výkonem zobrazovacího hardwaru porostou kvalita vizuálních efektů a možnosti inter, avšak vždy bude nejvíce záležet na schopnostech autora.

\section{Velikost 64KB}
Velikost 64KB se může zdát pro aplikaci zobrazující grafiku velmi málo, když i velmi jednoduché programy mohou po kompilaci zabírat velikost několika megabajtů.
Toto se často týká vysokoúrovňových jazyků a vysokoúrovňových programovacích prostředků.
Tyto vysokoúrovňové nástroje poskytují programátorský komfort, ale obvykle nedbají na vý\-sled\-nou velikost aplikace a ani na její rychlost.
V podstatě je mezi kódem programátora a výslednými instrukcemi procesoru několik mezivrstev.
V samotných aplikacích se tak vyskytují spousty nevyužitého kódu a statických dat.
Když se řekne počítačová grafika, většině uživatelů se vybaví počítačové hry a ty běžně zabírají gigabajty dat, což je i sto tisíckrát více než obvyklá velikost inter.
Toto je jedna z nejdůležitějších (ne-li nejdůležitější) charakteristik intra.

Základem malé velikosti inter je tedy odstranění všech nevyužitých částí.
Toho lze dosáhnou vhodným (nízkoúrovňovým) programovacím jazykem jako je jazyk C nebo assembler.
Dále vhodným nastavením překladače a vhodným programátorským stylem.
Další velkou úsporu místa lze dosáhnout generováním grafického obsahu.
Místo, aby se data jako textury a modely ukládaly do konstantních dat, uloží se jen způsob jejich vytvoření.
Poslední možností je výslednou aplikaci zkomprimovat. 

\section{Příběh}
Od začátku jsem chtěl intro, které zobrazuje přírodu.
Nechtěl jsem intro, které je příliš abstraktní a mechanické.
Intro, ve kterém se vyskytují nereálné mechanické objekty.
Proto jsem zvolil mírný motiv - přírodu.
V intru jsou nakonec čtyři scény.
První z nich je vysokohorská scéna se zasněženými vrcholky štítů.
Hory jsou snímány z ptačího pohledu a jsou nasvíceny ranním sluncem.
Následuje přímořská scéna.
V ní je mírná pahorkatina a mořská hladina.
Scéna je osvícena zapadajícím sluncem.
Ve scéně se kamera ponoří do stráně.
Následuje scéna s tunelem, který směřuje dolů, do kopce.
Tunel je zarostlý a vyskytuje se v něm pavučina.
Na konci je voda.
Poslední scéna je v jeskyni.
Jeskyně je z části zatopena.
Do jeskyně proniká voda v podobě několika vodopádů.
Jsou zde zavěšeny tři barevné koberce, pod kterými jsou rozházeny bedny.
Scéna ke konci intra vybouchne.


